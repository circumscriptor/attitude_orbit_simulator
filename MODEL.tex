\documentclass[11pt, a4paper]{article}
\usepackage[utf8]{inputenc}
\usepackage[T1]{fontenc}
\usepackage{amsmath}
\usepackage{amssymb}
\usepackage{bm}
\usepackage{geometry}
\usepackage{listings}
\usepackage{xcolor}
\usepackage{fancyhdr}
\usepackage{titling}

\geometry{
	left=10mm,
	right=10mm,
	top=15mm,
	bottom=15mm,
}

\lstset{
	language=C++,
	basicstyle=\ttfamily\small,
	keywordstyle=\color{blue}\bfseries,
	stringstyle=\color{red},
	commentstyle=\color{green!50!black},
	numbers=none,
	frame=single,
	breaklines=true,
	tabsize=4
}

\title{CubeSat Attitude Control Simulation}
\author{AOCS Team}
\date{\today}

\fancyhf{}
\fancyhead[C]{\MakeUppercase{\thetitle}}
\fancyfoot[L]{\theauthor}
\fancyfoot[C]{\thepage}
\fancyfoot[R]{\thedate}
\pagestyle{fancy}

\begin{document}

	\section{The State Vector $\mathbf{Y}$}

	We define the complete state of the system as a single vector $\mathbf{Y}$:

	\begin{equation}
		\mathbf{Y} = [\mathbf{r}, \mathbf{v}, \mathbf{q}, \bm{\omega}, \mathbf{M}_{irr}]^T
	\end{equation}

	Where:
	\begin{itemize}
		\item $\mathbf{r}$: Position vector in the \textbf{ECI} (Inertial) frame ($m$).
		\item $\mathbf{v}$: Velocity vector in the \textbf{ECI} frame ($m/s$).
		\item $\mathbf{q}$: Attitude quaternion (ECI $\to$ Body frame), defined as $[w, x, y, z]$ with unit norm.
		\item $\bm{\omega}$: Angular velocity vector in the \textbf{Body} frame ($rad/s$).
		\item $\mathbf{M}_{irr}$: A vector containing the scalar irreversible magnetization for each hysteresis rod ($A/m$).
	\end{itemize}

	\section{The Differential Equation $\frac{d\mathbf{Y}}{dt}$}
	\label{sec:diff_eq}

	The system of first-order differential equations governing the state evolution is:

	\begin{equation}
		\frac{d\mathbf{Y}}{dt} = \begin{bmatrix}
			\dot{\mathbf{r}} \\
			\dot{\mathbf{v}} \\
			\dot{\mathbf{q}} \\
			\dot{\bm{\omega}} \\
			\dot{\mathbf{M}}_{irr}
		\end{bmatrix} = \begin{bmatrix}
			\mathbf{v} \\
			\mathbf{g}_{\text{total}}(\mathbf{r}, t) \\
			\frac{1}{2} \mathbf{q} \otimes [0, \omega_x, \omega_y, \omega_z]^T \\
			\mathbf{I}^{-1} \left( \bm{\tau}_{total} - \bm{\omega} \times (\mathbf{I}\bm{\omega}) \right) \\
			\mathbf{f}_{hyst}(\mathbf{B}_{body}, \dot{\mathbf{B}}_{body}, \mathbf{M}_{irr})
		\end{bmatrix}
	\end{equation}

	\subsection{Gravitational Acceleration}
	The term $\mathbf{g}_{\text{total}}(\mathbf{r}, t)$ represents the total gravitational acceleration vector in the ECI frame. It is computed via the \texttt{GeographicLib::GravityModel}, which evaluates the gradient of the Earth's geopotential $V$:

	\begin{equation}
		\mathbf{g}_{\text{total}} = \nabla V(r, \theta, \lambda)
	\end{equation}

	The potential $V$ is defined by a spherical harmonic expansion (EGM2008 model):
	\begin{equation}
		V(r, \theta, \lambda) = \frac{GM}{r} \left[ 1 + \sum_{n=2}^{N} \left(\frac{a}{r}\right)^n \sum_{m=0}^{n} (C_{nm} \cos m\lambda + S_{nm} \sin m\lambda) P_{nm}(\cos \theta) \right]
	\end{equation}

	This naturally includes the central Newtonian term ($n=0$), the $J_2$ oblateness perturbation ($n=2, m=0$), and higher-order zonal and tesseral harmonics up to degree $N$.

	\noindent \textit{Note: The term $\mathbf{f}_{hyst}$ represents the complex nonlinear evolution of the hysteresis rods, detailed in Section 3.5.}

	\subsection{Quaternion Normalization}
	The quaternion derivative $\dot{\mathbf{q}} = \frac{1}{2} \mathbf{q} \otimes [0, \omega_x, \omega_y, \omega_z]^T$ is exact, but numerical integration causes quaternions to drift from unit norm due to accumulated roundoff errors. To maintain the constraint $\|\mathbf{q}\| = 1$, the quaternion is explicitly normalized at regular intervals during integration:

	\begin{equation}
		\mathbf{q}_{\text{normalized}} = \frac{\mathbf{q}}{\|\mathbf{q}\|}
	\end{equation}

	This is performed every $N$ integration steps (typically $N = 10$) to prevent numerical drift while minimizing computational overhead.

	\section{Simulation Workflow $f(t, \mathbf{Y})$}

	\subsection{Time and Frame Setup}
	We must bridge the gap between the Inertial frame (ECI) and the Earth-Fixed frame (ECEF/Geodetic).

	\begin{enumerate}
		\item \textbf{Calculate Earth Rotation Angle:}\\
		Compute the rotation angle $\theta_{rot} = \omega_{\oplus} \cdot t$, where $\omega_{\oplus} = 7.2921159 \times 10^{-5}$ rad/s is Earth's rotation rate.

		\item \textbf{Construct Rotation Matrix $\mathbf{R}_{ECEF}^{ECI}$:}
		\begin{equation}
			\mathbf{R}_{ECEF}^{ECI} = \begin{bmatrix}
				\cos\theta_{rot} & -\sin\theta_{rot} & 0 \\
				\sin\theta_{rot} & \cos\theta_{rot} & 0 \\
				0 & 0 & 1
			\end{bmatrix}
		\end{equation}

		\item \textbf{Position Conversion:}
		\begin{equation}
			\mathbf{r}_{ECEF} = (\mathbf{R}_{ECEF}^{ECI})^T \mathbf{r}_{ECI}
		\end{equation}
	\end{enumerate}

	\subsection{GeographicLib Coordinates}
	Use the \texttt{Geocentric} class to obtain Geodetic coordinates and the local frame rotation matrix.

	\begin{lstlisting}
GeographicLib::Geocentric earth(Constants::WGS84_a(), Constants::WGS84_f());
double lat, lon, h;
std::vector<double> M;
// Computes lat/lon/h AND the rotation from ENU to ECEF
earth.Reverse(r_ecef.x, r_ecef.y, r_ecef.z, lat, lon, h, M);
	\end{lstlisting}

	\begin{itemize}
		\item \textbf{Input:} $\mathbf{r}_{ECEF}$.
		\item \textbf{Output:} Latitude ($\phi$), Longitude ($\lambda$), Height ($h$).
		\item \textbf{Output:} $\mathbf{R}_{ENU}^{ECEF}$ as a $3\times3$ matrix.
	\end{itemize}

	\subsection{Environmental Vectors}

	\paragraph{A. Gravitational Acceleration (GravityModel)}
	The gravity vector is obtained in the Local Tangent Plane (ENU) and includes all perturbations:
	\begin{lstlisting}
// Returns total acceleration g (including J2, higher harmonics)
grav.Gravity(lat, lon, h, gx, gy, gz);
Vector3d g_ENU(gx, gy, gz);
	\end{lstlisting}
	Transform to Inertial Frame (ECI):
	\begin{equation}
		\mathbf{g}_{\text{total}}(\mathbf{r}, t) = \mathbf{R}_{ECEF}^{ECI} \cdot \mathbf{R}_{ENU}^{ECEF} \cdot \mathbf{g}_{ENU}
	\end{equation}

	\paragraph{B. Magnetic Field (MagneticModel)}
	The magnetic field vector is obtained in ENU coordinates:
	\begin{lstlisting}
mag(year, lat, lon, h, Bx, By, Bz); // Output in nT
Vector3d B_ENU(Bx * 1e-9, By * 1e-9, Bz * 1e-9); // Convert to Tesla
	\end{lstlisting}
	Transform to ECI frame:
	\begin{equation}
		\mathbf{B}_{ECI} = \mathbf{R}_{ECEF}^{ECI} \cdot \mathbf{R}_{ENU}^{ECEF} \cdot \mathbf{B}_{ENU}
	\end{equation}

	\subsection{Magnetic Field Derivative ($\dot{\mathbf{B}}_{ECI}$)}

	The time derivative of the magnetic field in the ECI frame is computed using a numerical spatial gradient approach. This captures the \textbf{material derivative} (also called the total derivative) following the satellite's trajectory:

	\begin{equation}
		\frac{D\mathbf{B}_{ECI}}{Dt} = \frac{\partial \mathbf{B}_{ECI}}{\partial t} + (\mathbf{v}_{ECI} \cdot \nabla)\mathbf{B}_{ECI}
	\end{equation}

	where the first term represents secular variation (negligible on orbital timescales, $\sim 10^{-9}$ T/s) and the second term represents spatial variation due to orbital motion (dominant, $\sim 10^{-6}$ T/s).

	\textbf{Numerical Implementation:}
	Using a finite difference approximation with a fixed timestep $\Delta t = 0.01$ s:
	\begin{equation}
		\frac{D\mathbf{B}_{ECI}}{Dt} \approx \frac{\mathbf{B}_{ECI}(\mathbf{r} + \mathbf{v}\Delta t, t + \Delta t) - \mathbf{B}_{ECI}(\mathbf{r}, t)}{\Delta t}
	\end{equation}

	This automatically captures the spatial gradient term, which is the dominant contribution. The fixed timestep $\Delta t$ is chosen to balance numerical accuracy (avoiding cancellation errors) with approximation quality (capturing local field curvature). For LEO velocities ($\sim 7.5$ km/s), this corresponds to a spatial displacement of $\sim 75$ m.

	\textbf{Important Notes:}
	\begin{itemize}
		\item The timestep $\Delta t$ is a \textit{numerical differentiation parameter}, independent of the ODE integrator's adaptive timestep.
		\item This approach works correctly with variable-step integrators (RK45, Dormand-Prince).
		\item The secular variation $\partial \mathbf{B}/\partial t$ from the WMM model ($\sim$nT/year) is negligible compared to the orbital motion term and is not explicitly computed.
	\end{itemize}

	\subsection{Magnetic Field Rate in Body Frame}

	For hysteresis rod dynamics, we need the time derivative of $\mathbf{B}$ as measured in the \textit{body-fixed} frame. This has two contributions:

	\begin{equation}
		\frac{d\mathbf{B}_{body}}{dt} = \underbrace{\mathbf{R}_{ECI}^{Body}(\mathbf{q}) \frac{D\mathbf{B}_{ECI}}{Dt}}_{\text{Orbital Motion (Dominant)}} - \underbrace{\bm{\omega} \times \mathbf{B}_{body}}_{\text{Body Rotation (Secondary)}}
	\end{equation}

	where:
	\begin{itemize}
		\item The first term transforms the material derivative from ECI to the body frame, representing how the field changes as the satellite moves through Earth's magnetic field ($\sim 10^{-6}$ T/s).
		\item The second term arises from the transport theorem (time derivative in a rotating reference frame), representing the apparent field change due to spacecraft rotation ($\sim 10^{-7}$ T/s for typical tumbling rates).
		\item $\mathbf{R}_{ECI}^{Body}(\mathbf{q})$ is the rotation matrix derived from the attitude quaternion $\mathbf{q}$.
	\end{itemize}

	\textbf{Physical Interpretation:} Even when the spacecraft is not rotating ($\bm{\omega} = 0$), the rods experience a time-varying magnetic field due to orbital motion at $\sim 7.5$ km/s through Earth's spatially-varying field. This is the primary mechanism for hysteresis damping.

	\subsection{Hysteresis Dynamics ($\dot{\mathbf{M}}_{irr}$)}
	The function $\mathbf{f}_{hyst}$ mentioned in Section~\ref{sec:diff_eq} is computed here. For each rod $i$ aligned with the body axis unit vector $\mathbf{u}_i$:

	\begin{enumerate}
		\item \textbf{Project Field and Rate onto Rod:}
		\begin{align}
			H_i &= \frac{1}{\mu_0} (\mathbf{B}_{body} \cdot \mathbf{u}_i) \\
			\dot{H}_i &= \frac{1}{\mu_0} \left( \frac{d\mathbf{B}_{body}}{dt} \cdot \mathbf{u}_i \right)
		\end{align}

		\item \textbf{Calculate Theoretical Anhysteretic Magnetization ($M_{an}$):}
		\begin{align}
			H_{eff} &= H_i + \alpha M_{irr, i} \\
			M_{an} &= M_s \left( \coth\left(\frac{H_{eff}}{a}\right) - \frac{a}{H_{eff}} \right)
		\end{align}

		\item \textbf{Calculate Derivative (The Hysteresis Differential Equation):}\\
		The classic Jiles-Atherton model gives $dM/dH$. We require $dM/dt$ for state integration:
		\begin{equation}
			\frac{dM_{irr, i}}{dH} = \frac{M_{an} - M_{irr, i}}{k \delta} \quad \text{where } \delta = \text{sign}(\dot{H}_i)
		\end{equation}
		Applying the chain rule:
		\begin{equation}
			\frac{dM_{irr, i}}{dt} = \left( \frac{M_{an} - M_{irr, i}}{k \cdot \text{sign}(\dot{H}_i)} \right) \cdot \dot{H}_i
		\end{equation}

		\textbf{Singularity Handling:} When $\dot{H}_i \to 0$, the sign function creates a discontinuity. To prevent integrator instability:
		\begin{itemize}
			\item Use a smooth approximation: $\text{sign}(x) \approx \tanh(\beta x)$ with large $\beta$.
			\item Or implement a dead zone: if $|\dot{H}_i| < \epsilon$, set $dM_{irr,i}/dt = 0$.
		\end{itemize}
	\end{enumerate}

	\subsection{Torque Dynamics ($\bm{\tau}_{total}$)}
	The total external torque acting on the spacecraft is the sum of magnetic and gravity gradient contributions:
	\begin{equation}
		\bm{\tau}_{total} = \bm{\tau}_{mag} + \bm{\tau}_{grad}
	\end{equation}

	\subsubsection{Magnetic Torque ($\bm{\tau}_{mag}$)}
	This is the interaction between the spacecraft's total dipole moment and the Earth's magnetic field. The dipole moment consists of the permanent magnet ($\mathbf{m}_{perm}$) and the hysteresis rods ($\mathbf{m}_{rods}$).

	\begin{equation}
		\mathbf{m}_{rods} = \sum_{i=1}^{N} \left( V_{rod} \cdot M_{total, i} \cdot \mathbf{u}_i \right)
	\end{equation}

	where the total magnetization $M_{total}$ includes both the irreversible state variable and the reversible (elastic) component, scaled by the coefficient $c$:

	\begin{equation}
		M_{total, i} = (1 - c)M_{irr, i} + cM_{an}(H_{eff, i})
	\end{equation}

	The total magnetic torque is then:
	\begin{equation}
		\bm{\tau}_{mag} = (\mathbf{m}_{perm} + \mathbf{m}_{rods}) \times \mathbf{B}_{body}
	\end{equation}

	\subsubsection{Gravity Gradient Torque ($\bm{\tau}_{grad}$)}
	In LEO, the Earth's gravitational field varies across the body of the spacecraft, creating a torque that tries to align the minimum moment of inertia with the nadir vector.

	\begin{equation}
		\bm{\tau}_{grad} = \frac{3\mu}{\lVert\mathbf{r}_{ECI}\rVert^5} \left( \mathbf{r}_{body} \times (\mathbf{I} \cdot \mathbf{r}_{body}) \right)
	\end{equation}

	where:
	\begin{itemize}
		\item $\mathbf{r}_{body} = \mathbf{R}_{ECI}^{Body}(\mathbf{q}) \cdot \mathbf{r}_{ECI}$ is the position vector in the Body frame.
		\item $\mathbf{I}$ is the spacecraft inertia tensor (diagonal for principal axes).
		\item $\mu = 3.986004418 \times 10^{14}$ m$^3$/s$^2$ is Earth's standard gravitational parameter.
	\end{itemize}

\end{document}
