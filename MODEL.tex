\documentclass[11pt, a4paper]{article}
\usepackage[utf8]{inputenc}
\usepackage[T1]{fontenc}
\usepackage{amsmath}
\usepackage{amssymb}
\usepackage{bm}
\usepackage{geometry}
\usepackage{listings}
\usepackage{xcolor}
\usepackage{fancyhdr}
\usepackage{titling}

\geometry{
	left=25mm,
	right=25mm,
	top=15mm,
	bottom=15mm,
}

\lstset{
	language=C++,
	basicstyle=\ttfamily\small,
	keywordstyle=\color{blue}\bfseries,
	stringstyle=\color{red},
	commentstyle=\color{green!50!black},
	numbers=none,
	frame=single,
	breaklines=true,
	tabsize=4
}

\title{CubeSat Attitude Control Simulation}
\author{AOCS Team}
\date{\today}

\fancyhf{}
\fancyhead[C]{\MakeUppercase{\thetitle}}
\fancyfoot[L]{\theauthor}
\fancyfoot[C]{\thepage}
\fancyfoot[R]{\thedate}
\pagestyle{fancy}

\begin{document}

	\section{The State Vector $\mathbf{Y}$}

	We define the complete state of the system as a single vector $\mathbf{Y}$:

	\begin{equation}
		\mathbf{Y} = [\mathbf{r}, \mathbf{v}, \mathbf{q}, \bm{\omega}, \mathbf{M}_{irr}]^T
	\end{equation}

	Where:
	\begin{itemize}
		\item $\mathbf{r}$: Position vector in the \textbf{ECI} frame ($m$).
		\item $\mathbf{v}$: Velocity vector in the \textbf{ECI} frame ($m/s$).
		\item $\mathbf{q}$: Attitude quaternion (ECI $\to$ Body), defined as $[w, x, y, z]$.
		\item $\bm{\omega}$: Angular velocity vector in the \textbf{Body} frame ($rad/s$).
		\item $\mathbf{M}_{irr}$: Scalar irreversible magnetization for each hysteresis rod ($A/m$).
	\end{itemize}

	\section{The Differential Equation $\frac{d\mathbf{Y}}{dt}$}
	\label{sec:diff_eq}

	The system of first-order differential equations governing the state evolution is:

	\begin{equation}
		\frac{d\mathbf{Y}}{dt} = \begin{bmatrix}
			\dot{\mathbf{r}} \\
			\dot{\mathbf{v}} \\
			\dot{\mathbf{q}} \\
			\dot{\bm{\omega}} \\
			\dot{\mathbf{M}}_{irr}
		\end{bmatrix} = \begin{bmatrix}
			\mathbf{v} \\
			\mathbf{g}_{\text{total}}(\mathbf{r}, t) \\
			\frac{1}{2} \mathbf{q} \otimes [0, \omega_x, \omega_y, \omega_z]^T \\
			\mathbf{I}^{-1} \left( \bm{\tau}_{total} - \bm{\omega} \times (\mathbf{I}\bm{\omega}) \right) \\
			\mathbf{f}_{hyst}(\mathbf{B}_{body}, \dot{\mathbf{B}}_{body}, \mathbf{M}_{irr})
		\end{bmatrix}
	\end{equation}

	\subsection{Gravitational Acceleration}
	The acceleration $\mathbf{g}_{\text{total}}$ is defined as the negative gradient of the geopotential $V$:

	\begin{equation}
		\mathbf{g}_{\text{total}} = -\nabla V(r, \theta, \lambda)
	\end{equation}

	The potential $V$ follows the EGM2008 model. \texttt{GeographicLib::GravityModel} returns the acceleration vector directly, which includes the central mass term and all spherical harmonic perturbations (zonal, tesseral, and sectoral).

	\subsection{Quaternion Normalization}
	To maintain the constraint $\|\mathbf{q}\| = 1$, the quaternion is explicitly normalized after \textbf{every successful integration step}:
	\begin{equation}
		\mathbf{q}_{next} = \frac{\mathbf{q} + \dot{\mathbf{q}}\Delta t}{\|\mathbf{q} + \dot{\mathbf{q}}\Delta t\|}
	\end{equation}
	Frequent normalization prevents the accumulation of errors that could otherwise lead to unphysical rotations or "jumps" that destabilize adaptive-step solvers.

	\section{Simulation Workflow $f(t, \mathbf{Y})$}

	\subsection{Time and Frame Setup}
	To synchronize with real-world observations (e.g., TLEs), we must account for the Earth's initial orientation:
	\begin{enumerate}
		\item \textbf{Calculate Rotation Angle:}\\
		$\theta_{rot} = \theta_{GMST,0} + \omega_{\oplus} \cdot t$, where $\theta_{GMST,0}$ is the Greenwich Mean Sidereal Time at $t=0$.
		\item \textbf{Rotation Matrix $\mathbf{R}_{ECEF}^{ECI}$:} Standard Z-axis rotation using $\theta_{rot}$.
		\item \textbf{Position Conversion:} $\mathbf{r}_{ECEF} = (\mathbf{R}_{ECEF}^{ECI})^T \mathbf{r}_{ECI}$.
	\end{enumerate}

	\subsection{GeographicLib and Environmental Vectors}
	The \texttt{GravityModel} and \texttt{MagneticModel} return values in the Local Tangent Plane (ENU). These are rotated back to ECI for the ODE:
	\begin{align}
		\mathbf{g}_{ECI} &= \mathbf{R}_{ECEF}^{ECI} \cdot \mathbf{R}_{ENU}^{ECEF} \cdot \mathbf{g}_{ENU} \\
		\mathbf{B}_{ECI} &= \mathbf{R}_{ECEF}^{ECI} \cdot \mathbf{R}_{ENU}^{ECEF} \cdot \mathbf{B}_{ENU}
	\end{align}

	\subsection{Magnetic Field Derivative}
	The material derivative captures field changes due to orbital motion:
	\begin{equation}
		\frac{D\mathbf{B}_{ECI}}{Dt} \approx \frac{\mathbf{B}_{ECI}(\mathbf{r} + \mathbf{v}\Delta t, t + \Delta t) - \mathbf{B}_{ECI}(\mathbf{r}, t)}{\Delta t}
	\end{equation}
	In the Body frame, including the transport term (apparent change due to spacecraft rotation):
	\begin{equation}
		\frac{d\mathbf{B}_{body}}{dt} = \mathbf{R}_{ECI}^{Body}(\mathbf{q}) \frac{D\mathbf{B}_{ECI}}{Dt} - \bm{\omega} \times \mathbf{B}_{body}
	\end{equation}

	\subsection{Hysteresis Dynamics ($\dot{\mathbf{M}}_{irr}$)}
	For each rod $i$, we project the field $H_i$ and its rate $\dot{H}_i$ onto the rod's axis $\mathbf{u}_i$. The Jiles-Atherton evolution is:
	\begin{equation}
		\frac{dM_{irr, i}}{dt} = \dot{H}_i \cdot \frac{M_{an, i} - M_{irr, i}}{k \cdot \text{sign}(\dot{H}_i)} = \frac{M_{an, i} - M_{irr, i}}{k} |\dot{H}_i|
	\end{equation}

	\textbf{Physicality and Stability Constraints:}
	\begin{enumerate}
		\item \textbf{Directional Constraint:} Magnetization must always move towards the anhysteretic curve. If $(M_{an} - M_{irr})\dot{H}_i < 0$, we set $\dot{M}_{irr, i} = 0$ to prevent unphysical state excursions.
		\item \textbf{Smoothing:} The $\text{sign}(\dot{H}_i)$ function is approximated by $\tanh(\beta \dot{H}_i)$ with $\beta \approx 10^6$ to maintain continuity for the ODE solver.
		\item \textbf{Singularity:} If $|\dot{H}_i| < \epsilon$, then $\dot{M}_{irr, i} = 0$.
	\end{enumerate}

	\subsection{Torque Dynamics}
	\paragraph{Magnetic Torque:}
	Includes permanent magnets and the total rod magnetization (reversible + irreversible components):
	\begin{equation}
		\bm{\tau}_{mag} = \left(\mathbf{m}_{perm} + \sum [V_{rod, i} ((1-c)M_{irr, i} + cM_{an, i}) \mathbf{u}_i] \right) \times \mathbf{B}_{body}
	\end{equation}

	\paragraph{Gravity Gradient Torque:}
	\begin{equation}
		\bm{\tau}_{grad} = \frac{3\mu}{\|\mathbf{r}_{ECI}\|^5} \left( \mathbf{r}_{body} \times (\mathbf{I} \cdot \mathbf{r}_{body}) \right)
	\end{equation}

\end{document}
